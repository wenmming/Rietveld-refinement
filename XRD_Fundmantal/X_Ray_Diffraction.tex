\documentclass[a4paper,12pt]{article}

\usepackage{amsmath}
\usepackage{amssymb}
\usepackage[UTF8]{ctex}
\usepackage{graphicx}
\usepackage{hyperref}

\begin{document}
\begin{section}{X射线衍射强度}
由于实际晶体不具有理想的完整性,同时入射X射线也不可能完全平行和单色,因此测量晶体某一$(hkl)$晶面的某一位置时的衍射强度对于晶体结构的测定未必有意义。对于多晶衍射,通常所测量的是某一衍射面$(hkl)$的积分衍射强度$I_\Sigma$。对于取向完全随机的粉末试样,其衍射峰如图\ref{fig:integralIntensity}所示。图中的阴影部分表示积分强度。

\begin{figure}[htbp]
\centering
\includegraphics[width=200pt]{integralIntensity.pdf}
\caption{粉末衍射峰示意图,阴影面积为积分强度}
\label{fig:integralIntensity}
\end{figure}

多晶X射线衍射的积分强度表达式为:\[I_\Sigma=\dfrac{N^2e^4\lambda^3}{32\pi Rm_0^2c^4}I_0v\cdot M\cdot L_P\cdot F^2\cdot e^{-2B \sin^2\theta/\lambda^2}\cdot A^*(\theta)\cdot (PO)\]

式中:

\[\begin{array}{llll}
N & \text{单位体积内的晶胞数} & \lambda & \text{X射线波长}\\
R & \text{照相机(或衍射仪)半径} & M & \text{多重性因子}\\
e & \text{电子电荷量} & L_P & \text{洛伦兹偏振因子}\\
m_0 & \text{电子静质量} & F & \text{结构因子}\\
c & \text{光速} & B & \text{热学参数}\\
v & \text{受X射线照射的试样体积} & A^*(\theta) & \text{吸收因子$A(\theta)$的倒数}\\
I_0 & \text{入射X射线的强度} & (PO) & \text{择优取向因子}
\end{array}\]


令$K=\tfrac{N^2e^4\lambda^3}{32\pi Rm_0^2c^4}I_0v$,在同一实验室条件下$K$为常数,衍射的相对积分强度$I$的表达式为:\[I=M\cdot L_P\cdot F^2\cdot e^{-2B \sin^2\theta/\lambda^2}\cdot A^*(\theta)\cdot (PO)\]
\end{section}
\end{document}