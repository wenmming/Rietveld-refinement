\documentclass{article}

\usepackage{amsmath}
\usepackage{amssymb}
\usepackage[UTF8]{ctex}
\usepackage{hyperref}
\usepackage{graphicx}

\begin{document}
\begin{section}{温度因子}

温度因子(Thermal parameters),又称原子位移因子(Atomic displacement parameters),用来描述原子的热振动引起原子散射能力的减弱。

晶体中的原子在平衡位置附近不停的振动,并且振动幅度随着温度的增加而增大。原子热振动幅度越大,对X射线的散射能力就会越弱。德拜首先提出热运动对衍射强度的影响之后,沃勒又进行了进一步修正,给出了热振动对衍射强度减弱的系数为$e^{-2B\sin^2\theta/\lambda^2}$,其中$B$就是温度因子。

温度因子$B$与原子振幅$u$的关系为:$B=8\pi^2\overline{u^2}$。令$U=\overline{u^2}$,则有$B=8\pi U$,因此$U$也能够代表温度对原子散射能力的影响。$B$值一般认为在1.0 \AA$^2$左右:对于金属氧化物中结合比较紧密的原子,它的典型值约为0.5 \AA$^2$,然而对于有机分子等结构,典型值增加到3$\sim$5 \AA$^2$。对一个结构良好的化合物,其中原子的振动接近于各向同性,因此在XRD精修时主要考虑各向同性温度因子$B_\text{iso}/U_\text{iso}$就足够了。

加入温度对原子散射能力的影响之后,完整的结构因子可以表达为:
\[F=\sum_n f_n N_n \exp\{2\pi \mathrm{i} (hx+ky+lz)\}\exp(-2B\sin^2\theta/\lambda^2)\]

其中$N_n$表示原子在某个位置的占有率。

\end{section}
\end{document}